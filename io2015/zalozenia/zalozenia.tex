%%%%%%%%
\documentclass [11pt, a4paper, leqno] {article}
\usepackage [polish] {babel}
\usepackage {polski}
\usepackage [utf8] {inputenc}
\usepackage [T1] {fontenc}
\usepackage {indentfirst}
\usepackage {setspace}
\setstretch {1.15} %interlinia
\frenchspacing
\renewcommand {\thesection} {\arabic{section}.}
\renewcommand {\thesubsection} {\arabic{section}. \arabic{subsection}.}
\renewcommand {\thesubsubsection} {\arabic{section}. \arabic{subsection}.\arabic{subsubsection}.}
\textheight 620pt
\makeatletter
\renewcommand\@biblabel[1]{#1}
\makeatother
%%%%%%%%

\begin{document}
% STRONA TYTUŁOWA

\begin{center}
  \thispagestyle{empty} %usunięcie numeracji
  {\large Studencka Pracownia Inżynierii Oprogramowania} \\ [0.5cm]
	{\large Instytut Informatyki Uniwersytetu Wrocławskiego} \\ [6.0cm]

  {\large Rafał Florczak, Dominik Rabij} \\ [1.5cm]

	{\huge Dokumentacja oprogramowania} \\ [0.5cm]
  {\huge RECEPCJONISTA} \\ [1.5cm]

  {\large Założenia wstępne} \\ [0.5cm]

  \vfill
  
  %{\large Wrocław, \today} \\ [0.5cm]
  {\large Wrocław, 6 listopada 2015} \\ [0.5cm]
  {\large Wersja 0.3}
\end{center}

\newpage

% TABELKA WERSJI

\textit{Tabela 0.} Historia zmian dokonanych w dokumencie

\begin{center}
  \begin{tabular}{| l | l | l | l |}
    \hline
    \multicolumn{1}{|c|}{Data} & 
    \multicolumn{1}{|c|}{Numer wersji} &  
    \multicolumn{1}{|c|}{Opis} &
    \multicolumn{1}{|c|}{Autor} \\ \hline \hline
    2015-11-02 & 0.1 & Utworzenie dokumentu & Rafał Florczak \\ \hline
    2015-11-05 & 0.2 & Naniesienie poprawek & Dominik Rabij\\ \hline
    2015-11-06 & 0.3 & Naniesienie poprawek & Rafał Florczak\\ \hline
  \end{tabular}
\end{center}

\medskip

\tableofcontents

\newpage

\section{Wprowadzenie}

\subsection{Cel dokumentu}
\noindent
Celem niniejszego dokumentu jest przedstawienie i opis wymagań ogólnych oprogramowania wspomagającego prowadzenie rezerwacji miejsc hotelowych (dalej nazywanego ,,Recepcjonistą''). Szczegóły realizacji kolejnych kroków związanych z powstawaniem Recepcjonisty zostaną opisane w następnych dokumentach.

Ten dokument jest przeznaczony dla zleceniodawcy oraz osób wdrażających oprogramowanie.

\subsection{Ogólny cel powstania Recepcjonisty}
\noindent
Głównym celem Recepcjonisty jest ułatwienie procesu rezerwacji pokojów hotelowych przez klientów hotelu oraz potwierdzanie takich rezerwacji przez obsługę hotelową.

Dla klientów przewidziany jest specjalny formularz osadzony na stronie internetowej hotelu, który w prosty i przejrzysty sposób przeprowadzi ich przez proces rezerwacji pokojów.

Od strony administracyjnej Recepcjonista umożliwia potwierdzanie i odrzucanie rezerwacji klientów, dodawanie nowych pokojów, a także modyfikowanie i kasowanie już istniejących.

\section{Opis użytkowników}

\subsection{Klienci hotelu}
\noindent
Podstawowymi użytkownikami Recepcjonisty są klienci hotelu -- to dla nich przeznaczony jest formularz wspomniany w poprzednim punkcie.

Klienci mają możliwość zapoznania się z dostępnymi pokojami w zadanym przedziale czasu. Mogą również skontaktować się z obsługą hotelu za pośrednictwem odrębnego formularza kontaktowego.

Rezerwacja pokojów odbywa się przy użyciu formularza osadzonego na stronie internetowej hotelu. Rezerwowane pokoje będą wybrane na podstawie wyznaczonego przedziału czasu, liczby i klasy pokojów oraz szczegółów takich jak piętro czy liczba łazienek przypisanych do konkretnego pokoju.

\subsection{Pracownicy hotelu}
\noindent
Upoważnieni pracownicy hotelu mogą dodawać nowe pokoje oraz modyfikować i usuwać istniejące. Dodatkowo do ich obowiązków należy potwierdzanie oraz odrzucanie rezerwacji klientów. W celach weryfikacyjnych mają oni dostęp do wszystkich danych wprowadzonych do systemu podczas składania rezerwacji przez klienta.

%\section{Zaplecze sprzętowe i programowe}
%\subsection{Sprzęt}
%\noindent
%Do uruchomienia i sprawnego działania Recepcjonisty potrzebny jest komputer spełniający poniższą specyfikację: 
%\begin{itemize}
%\item procesor klasy Pentium II lub nowszy,
%\item pamięć RAM 4GB lub większa,
%\item dysk o pojemności 1TB lub większej.
%\end{itemize}

%\subsection{System}
%\noindent
%System operacyjny Debian w wersji 6 lub nowszej. Reszta zewnętrznego oprogramowania zostanie dostarczona wraz z %produktem końcowym.

\section{Harmonogram}
\noindent
Projekt został podzielony na następujące etapy:
\begin{itemize}
\item analiza,
\item projektowanie,
\item implementacja,
\item testowanie,
\item wdrażanie.
\end{itemize}

\subsection{Analiza}
\noindent
Etap analizy polega na szczegółowym opisaniu wymagań dotyczących projektu we współpracy z klientem. Po zakończeniu tej fazy określone będą też cele oprogramowania. Etap powinien zająć do 5 dni roboczych.

\subsection{Projektowanie}
\noindent
W tym etapie zostaną stworzone pierwsze szkice interfejsu graficznego aplikacji, opracowana będzie struktura bazy danych oraz podział programu na moduły. Na koniec dokonany zostanie też podział prac. Etap powinien zająć do 10 dni roboczych.

\subsection{Implementacja}
\noindent
Na podstawie projektów utworzonych w poprzednim etapie zespół programistów będzie tworzył kolejne moduły Recepcjonisty. Przewiduje się, że ten etap zajmie do 6 miesięcy.

\subsection{Testowanie}
\noindent
Oprogramowanie będzie poddawane dogłębnym testom. Sprawdzone zostaną możliwe scenariusze zachowań użytkowników oraz zgodność ze specyfikacją utworzoną w fazach analizy i projektowania. Testowanie będzie trwało nie dłużej niż miesiąc.

\subsection{Wdrażanie}
\noindent
Na tym etapie prace związane z implementacją i testowaniem powinny być już zakończone. Etap ten obejmuje instalację i konfigurację oprogramowania oraz szkolenia w zakresie jego obsługi. Oczekujemy, że potrwa to około tygodnia.

\end{document}