%%%%%%%%
\documentclass [11pt, a4paper, leqno] {article}
\usepackage [polish] {babel}
\usepackage {polski}
\usepackage [utf8] {inputenc}
\usepackage [T1] {fontenc}
\usepackage {indentfirst}
\usepackage {setspace}
\usepackage{amstext}
\setstretch {1.15} %interlinia
\frenchspacing
\renewcommand {\thesection} {\arabic{section}.}
\renewcommand {\thesubsection} {\arabic{section}. \arabic{subsection}.}
\renewcommand {\thesubsubsection} {\arabic{section}. \arabic{subsection}.\arabic{subsubsection}.}
\textheight 610pt
\makeatletter
\renewcommand\@biblabel[1]{#1}
\makeatother
\usepackage[a4paper,left=3.8cm,right=3.8cm,top=3.3cm,bottom=3cm]{geometry}
%%%%%%%%

\begin{document}
% STRONA TYTUŁOWA

\begin{center}
  \thispagestyle{empty} %usunięcie numeracji
  {\large Studencka Pracownia Inżynierii Oprogramowania} \\ [0.5cm]
	{\large Instytut Informatyki Uniwersytetu Wrocławskiego} \\ [6.0cm]

  {\large Rafał Florczak, Dominik Rabij} \\ [1.5cm]

	{\huge Dokumentacja oprogramowania} \\ [0.5cm]
  {\huge RECEPCJONISTA} \\ [1.5cm]

  {\large Zasoby konieczne do realizacji projektu oraz kosztorys} \\ [0.5cm]

  \vfill
  
  {\large Wrocław, \today} \\ [0.5cm]
  %{\large Wrocław, 6 listopada 2015} \\ [0.5cm]
  {\large Wersja 0.2}
\end{center}

\newpage

% TABELKA WERSJI

\textit{Tabela 0.} Historia zmian dokonanych w dokumencie

\begin{center}
  \begin{tabular}{| l | l | l | l |}
    \hline
    \multicolumn{1}{|c|}{Data} & 
    \multicolumn{1}{|c|}{Numer wersji} &  
    \multicolumn{1}{|c|}{Opis} &
    \multicolumn{1}{|c|}{Autor} \\ \hline \hline
    2015-12-02 & 0.1 & Utworzenie dokumentu & Dominik Rabij \\ \hline
    2015-12-16 & 0.2 & Naniesienie poprawek & Rafał Florczak \\ \hline
  \end{tabular}
\end{center}

\medskip

\tableofcontents

\newpage

\section{Wprowadzenie}

\subsection{Cel dokumentu}
\noindent
Celem niniejszego dokumentu jest przedstawienie zasobów niezbędnych do sprawnego zrealizowania Recepcjonisty oraz oszacowanie kosztów całego przedsięwzięcia. 

Dokument jest przeznaczony dla osób decyzyjnych po stronie klienta oraz dla zleceniobiorcy.

\section{Zasoby}

\subsection{Zasoby ludzkie}
\noindent
Osoby zaangażowane w tworzenie Recepcjonisty:
\begin{itemize}
\item kierownik do spraw sprzedaży,
\item kierownik projektu,
\item zespół programistów:
  \begin{itemize}
  \item doświadczony programista (lider),
  \item czterech stażystów,
  \end{itemize}
\item dwóch grafików,
\item pięciu testerów.
\end{itemize}

\subsection{Zasoby sprzętowe}
\noindent
Dla każdej osoby pracującej nad projektem potrzebna jest stacja robocza wraz z oprogramowaniem. Dodatkowo grafikom należy zapewnić dostęp do tabletów graficznych oraz dodatkowych monitorów.

\section{Kosztorys}
\noindent
Kosztorys został podzielony na części odpowiadające etapom tworzenia projektu, które zostały wydzielone w dokumencie \cite{zalozenia}. Trzeba również pamiętać o koszcie sprzętu i oprogramowania z p. 2.2.

\subsection{Analiza}
\noindent
W fazie analizy główną rolę odgrywa kierownik do spraw sprzedaży, gdyż to on kontaktuje się z klientem i tworzy dokument wymagań dotyczących oprogramowania.

\begin{equation}1\\\ 000 \text{ zł/dzień} \times 5 \text{ dni} = 5\\\ 000 \text{ zł} \end{equation}

\subsection{Projektowanie}
\noindent
Przy projektowaniu biorą udział kierownik projektu i lider zespołu programistów.

\begin{equation}2 \text{ osoby} \times 600 \text{ zł/dzień} \times 10 \text{ dni} = 12\\\ 000 \text{ zł} \end{equation}

\subsection{Implementacja}
\noindent
W czasie tej fazy opłacani są:
\begin{itemize}
\item kierownik projektu: 
\end{itemize}
\begin{equation}15\\\ 000 \text{ zł/miesiąc} \times 6 \text{ miesięcy} = 90\\\ 000 \text{ zł,} \end{equation}

\begin{itemize}
\item lider zespołu programistów: 
\end{itemize}
\begin{equation}12\\\ 000 \text{ zł/miesiąc} \times 6 \text{ miesięcy} = 72\\\ 000 \text{ zł,} \end{equation}

\begin{itemize}
\item pozostali programiści: 
\end{itemize}
\begin{equation}4 \text{ osoby} \times 2\\\ 000 \text{ zł/miesiąc} \times 6 \text{ miesięcy} = 48\\\ 000 \text{ zł,} \end{equation}

\begin{itemize}
\item graficy: 
\end{itemize}
\begin{equation}2 \text{ osoby} \times 6\\\ 000 \text{ zł/miesiąc} \times 6 \text{ miesięcy} = 72\\\ 000 \text{ zł.} \end{equation}

\subsection{Testowanie}
\noindent
W czasie testowania pracują:

\begin{itemize}
\item kierownik projektu: 
\end{itemize}
\begin{equation} 15\\\ 000 \text{ zł/miesiąc} \times 1 \text{ miesiąc} = 15\\\ 000 \text{ zł,} \end{equation}

\begin{itemize}
\item testerzy: 
\end{itemize}
\begin{equation}5 \text{ osób} \times 5\\\ 000 \text{ zł/miesiąc} \times 1 \text{ miesiąc} = 25\\\ 000 \text{ zł,} \end{equation}

\begin{itemize}
\item główny programista: 
\end{itemize}
\begin{equation}12\\\ 000 \text{ zł/miesiąc} \times 1 \text{ miesiąc} = 12\\\ 000 \text{ zł.} \end{equation}

\subsection{Wdrażanie}
\noindent
Przy wdrażaniu biorą udział:

\begin{itemize}
\item kierownik do spraw sprzedaży: 
\end{itemize}
\begin{equation}1\\\ 000 \text{ zł/dzień} \times 5 \text{ dni} = 5\\\ 000 \text{ zł,} \end{equation}

\begin{itemize}
\item kierownik projektu: 
\end{itemize}
\begin{equation}5\\\ 000 \text{ zł/dzień} \times 5 \text{ dni} = 25\\\ 000 \text{ zł.} \end{equation}

\subsection{Sprzęt}
\noindent
\begin{equation}14 \text{ stacji roboczych} \times 5\\\ 000 \text{ zł} + 2 \text{ tablety graficzne} \times 5\\\ 000 = 80\\\ 000 \text{ zł} \end{equation}

\subsection{Podsumowanie}
\noindent
Po zsumowaniu powyższych części składowych otrzymujemy:
\begin{equation}
\sum\limits_{i=1}^{12} (i) = 461\\\ 000 \text{ zł}
\end{equation}

\newpage
\addcontentsline{toc}{section}{Literatura}
\begin{thebibliography}{2}
	\bibitem{zalozenia} Florczak R., Rabij D.: \emph{Dokumentacja oprogramowania Recepcjonista. Założenia wstępne.} Wrocław, Studencka Pracownia Inżynierii Oprogramowania, Instytut Informatyki Uniwersytetu Wrocławskiego~2015.
\end{thebibliography}
\end{document}