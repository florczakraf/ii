%%%%%%%%
\documentclass [11pt, a4paper, leqno] {article}
\usepackage [polish] {babel}
\usepackage {polski}
\usepackage [utf8] {inputenc}
\usepackage [T1] {fontenc}
\usepackage {indentfirst}
\usepackage {setspace}
\usepackage {amstext}
\usepackage {amsfonts}
\setstretch {1.15} %interlinia
\frenchspacing
\renewcommand {\thesection} {\arabic{section}.}
\renewcommand {\thesubsection} {\arabic{section}. \arabic{subsection}.}
\renewcommand {\thesubsubsection} {\arabic{section}. \arabic{subsection}. \arabic{subsubsection}.}
\textheight 610pt
\makeatletter
\renewcommand\@biblabel[1]{#1}
\makeatother
\usepackage[a4paper,left=3.8cm,right=3.8cm,top=3.3cm,bottom=3cm]{geometry}
%%%%%%%%

\begin{document}
% STRONA TYTUŁOWA

\begin{center}
  \thispagestyle{empty} %usunięcie numeracji
  {\large Studencka Pracownia Inżynierii Oprogramowania} \\ [0.5cm]
	{\large Instytut Informatyki Uniwersytetu Wrocławskiego} \\ [6.0cm]

  {\large Rafał Florczak, Dominik Rabij} \\ [1.5cm]

	{\huge Dokumentacja oprogramowania} \\ [0.5cm]
  {\huge RECEPCJONISTA} \\ [1.5cm]

  {\large Słownik} \\ [0.5cm]

  \vfill
  
  %{\large Wrocław, \today} \\ [0.5cm]
  {\large Wrocław, 16 stycznia 2016} \\ [0.5cm]
  {\large Wersja 0.1}
\end{center}

\newpage

% TABELKA WERSJI

\textit{Tabela 0.} Historia zmian dokonanych w dokumencie

\begin{center}
  \begin{tabular}{| l | l | l | l |}
    \hline
    \multicolumn{1}{|c|}{Data} & 
    \multicolumn{1}{|c|}{Numer wersji} &  
    \multicolumn{1}{|c|}{Opis} &
    \multicolumn{1}{|c|}{Autor} \\ \hline \hline
    2016-01-16 & 0.1 & Utworzenie dokumentu & Dominik Rabij \\ \hline
  \end{tabular}
\end{center}

%\medskip

%\tableofcontents

\newpage

\section{Słownik}
\begin{description}
\item[access] --- pole określające poziom uprawnień użytkownika. Może przyjmować następujące wartości: normal, moderator, admin.

\item[admin] --- poziom uprawnień bez ograniczeń.

\item[bathrooms] --- pole zawierające liczbę łazienek w danym pokoju.

\item[baza danych] --- system przechowujący dane w formie tabel w zgodności z określonymi zasadami. 

\item[booked] --- rezerwacja, która nie została jeszcze potwierdzona przez uprawnionego pracownika hotelu.

\item[booking$_{-}$id] --- pole z unikalnym numerem identyfikacyjnym rezerwacji.

\item[bookings] --- tabela zawierająca informacje o rezerwacjach.

\item[canceled] --- rezerwacja, która została anulowana.

\item[date] --- data.

\item[description] --- pole zawierające opis pokoju, na przykład informacje o dodatkowym wyposażeniu, czy balkonie.

\item[double$_{-}$beds] --- pole określające liczbę łóżek dwuosobowych w pokoju.

\item[end$_{-}$date] --- pole z datą zakończenia rezerwacji pokoju.

\item[enumerable] --- typ wyliczeniowy.

\item[finished] --- rezerwacja, która zakończyła się powodzeniem.

\item[first$_{-}$name] --- pole zawierające imię użytkownika.

\item[floor] --- pole przechowujące numer piętra, na którym znajduje się dany pokój.

\item[implementacja] --- proces pisania kodu źródłowego programu.

\item[integer] --- liczba całkowita.

\item[last$_{-}$name] --- pole zawierające nazwisko użytkownika.

\item[moderator] --- poziom uprawnień pozwalający użytkownikowi na zarządzanie rezerwacjami oraz pokojami.

\item[normal] --- poziom uprawnień użytkownika, który pozwala na składanie rezerwacji. Umożliwia także przeglądanie i modyfikację istniejących rezerwacji, których twórcą jest użytkownik z tymi uprawnieniami.

\item[password] --- pole z posolonym hasłem użytkownika.

\item[path] --- pole przechowujące nazwy bezwzględne plików ze zdjęciami poszczególnych pokojów.

\item[phone] --- pole zawierające numer telefonu użytkownika.

\item[picture$_{-}$id] --- pole z unikalnym numerem identyfikacyjnym obrazka.

\item[pictures] --- tabela z informacjami o obrazkach wykorzystywanych w interfejsie graficznym Recepcjonisty.

\item[pole] --- najmniejsza część wiersza tabeli, która przechowuje pojedynczą daną.

\item[room$_{-}$id] --- pole z unikalnym numerem identyfikacyjnym pokoju.

\item[room$_{-}$number] --- pole z numerem pokoju.

\item[rooms] --- tabela z informacjami o pokojach hotelowych.

\item[single$_{-}$beds] --- pole określające liczbę łóżek jednoosobowych w pokoju.

\item[solenie] --- proces dodawania losowych danych do hasła w celu poprawy bezpieczeństwa.

\item[start$_{-}$date] --- pole zawierające datę rozpoczęcia rezerwacji pokoju.

\item[state] --- pole określające stan rezerwacji. Może przyjmować następujące wartości: booked, verified, finished, canceled.

\item[string] --- napis.

\item[system zarządzania bazą danych] --- system komunikujący się z bazą danych, który służy między innymi do odczytu i modyfikacji jej zawartości.

\item[user$_{-}$id] --- pole z unikalnym numerem identyfikacyjnym użytkownika systemu.

\item[users] --- tabela z danymi użytkowników.

\item[użytkownik] --- każda osoba korzystająca z Recepcjonisty.

\item[verified] --- rezerwacja potwierdzona przez uprawnionego pracownika hotelu.

\end{description}
\end{document}